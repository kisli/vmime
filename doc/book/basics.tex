\chapter{Basics}

% ============================================================================
\section{Reference counting}

\subsection{Introduction} % --------------------------------------------------

Since version 0.7.2cvs, VMime use smart pointers to simplify memory
management. VMime's smart pointer implementation relies on
RAII\footnote{Ressource Allocation is Initialisation} so that we do not need
to bother with deleting an object (freeing memory) when it is not used
anymore.

There are two possibilities for owning a reference to an object. We can own a
strong reference to an object: as long as we keep this reference, the object
is not destroyed. Or we can own a weak reference to the object: the object can
be destroyed if nobody owns a strong reference to it, in which case the weak
reference becomes invalid.

An object is destroyed as soon as the last strong reference to it is released.
At the same tine, all weak references (if any) are automatically set to point
to \vnull.

In VMime, these two types of references are known as {\vcode vmime::ref}
and {\vcode vmime::weak\_ref}, respectively.

\subsection{Instanciating reference-counted objects} % -----------------------

In VMime, all objects that support reference counting inherit from the
{\vcode vmime::object} class, which is responsible for
incrementing/decrementing the counter and managing the object's life cycle.
If you want to create a smart pointer to a new object instance, you should
use the function {\vcode vmime::create} instead of the {\vcode new}
operator.

\begin{lstlisting}[caption={Smarts pointers and creating objects}]
class myObject : public vmime::object
{
public:

   myObject(const vmime::string& name)
      : m_name(name)
   {
   }

   void sayHello()
   {
      std::cout << "Hello " << m_name << std::endl;
   }

private:

   vmime::string m_name;
};

int main()
{
   vmime::ref <myObject> obj =
      vmime::create <myObject>("world");

   obj->sayHello();

   return 0;

} // Here, 'obj' gets automatically destroyed
\end{lstlisting}

\subsection{Using smart pointers} % ------------------------------------------

Smart pointers are copiable, assignable and comparable. You can use them like
you would use normal ("raw") C++ pointers (eg. you can write
\lstinline{!ptr, ptr != NULL, ptr->method(), *ptr}...).

Type safety is also guaranteed, and you can type cast smart pointers using
the {\vcode staticCast()}, {\vcode dynamicCast()} and {\vcode constCast()}
equivalents on {\vcode vmime::ref} and {\vcode vmime::weak\_ref} objects:

\begin{lstlisting}[caption={Casting smart pointers}]
class myBase : public vmime::object { }
class myObject : public myBase { }

vmime::ref <myObject> obj = vmime::create <myObject>();

// Implicit downcast
vmime::ref <myBase> base = obj;

// Explicit upcast
vmime::ref <myObject> obj2 = base.dynamicCast <myObject>();
\end{lstlisting}

Weak references are used to resolve reference cycles (an object which refers
directly or indirectly to itself). The following example illustrates a
typical problem of reference counting:

\begin{lstlisting}
class parent : public vmime::object
{
public:

   void createChild(vmime::ref <child> c)
   {
      m_child = c;
   }

private:

   vmime::ref <child> m_child;
};

class child : public vmime::object
{
public:

   child(vmime::ref <parent> p)
      : m_parent(p)
   {
   }

private:

   vmime::ref <parent> m_parent;
};

int main()
{
   vmime::ref <parent> p = vmime::create <parent>();
   vmime::ref <child> c = vmime::create <child>();

   p->setChild(c);
}
\end{lstlisting}

In this example, neither {\vcode p} nor {\vcode c} will be deleted when
exiting {\vcode main()}. That's because {\vcode p} indirectly points to itself
{\em via} {\vcode c}, and {\em vice versa}. The solution is to use a weak
reference to the parent:

\begin{lstlisting}
vmime::weak_ref <parent> m_parent;
\end{lstlisting}

The decision to make the parent or the child a weak reference is purely
semantic, and it depends on the context and the relationships between the
objects. Note that when the parent is deleted, the {\vcode m\_parent} member
of the child points to \vnull.

More information about reference counting can be found on
Wikipedia\footnote{http://en.wikipedia.org/wiki/Reference\_counting}.

% ============================================================================
\section{Error handling}

In VMime, error handling is exclusively based on exceptions, there is no error
codes, or things like that.

VMime code may throw exceptions in many different situations: an unexpected
error occured, an operation is not supported, etc. You should catch them if
you want to report failures to the user. This is also useful when debugging
your program.

VMime exceptions support chaining: an exception can be encapsulated into
another exception to hide implementation details. The function
{\vcode exception::other()} returns the next exception in the chain,
or \vnull.

Following is an example code for catching VMime exceptions and writing error
messages to the console:

\begin{lstlisting}[caption={Catching VMime exceptions}]
std::ostream& operator<<(std::ostream& os, const vmime::exception& e)
{
   os << "* vmime::exceptions::" << e.name() << std::endl;
   os << "    what = " << e.what() << std::endl;

   // Recursively print all encapsuled exceptions
   if (e.other() != NULL)
      os << *e.other();

   return os;
}

...

try
{
   // ...some call to VMime...
}
catch (vmime::exception& e)
{
   std::cerr << e;         // VMime exception
}
catch (std::exception& e)
{
   std::cerr << e.what();  // standard exception
}
\end{lstlisting}

Read the source of {\vexample example6} if yo want to see a more complete
example of using VMime exceptions (such as getting more detailed information
by using specialized classes of {\vcode vmime::exception}).


% ============================================================================
\section{Basic objects}

\subsection{The {\vcode component} class} % ----------------------------------

In VMime, all the components of a message inherit from the same class
{\vcode component}. This includes the message itself (classes {\vcode message}
and {\vcode bodyPart}), the header, the header fields and the value of each
header field, the body and all the parts in the message.

The class component provide a common interface for parsing or generating all
these components (methods {\vcode parse()} and {\vcode generate()}). It also
provides additional functions to get some information about the parsing
process or the structure (methods {\vcode getParsedOffset()},
{\vcode getParsedLength()} and {\vcode getChildComponents()}).

Vmime also provides a set of classes corresponding to the basic types found
in a message; for example a mailbox, a mailbox list, date/time information,
media type, etc. They all inherit from {\vcode component} too.

\subsection{Date and time} % -------------------------------------------------

Date and time are used in several places in VMime, particularly in header
fields (Date, Received, ...). VMime fully supports RFC-2822's date and time
specification. The object {\vcode vmime::datetime} is used to manipulate date
and time information, and to parse/generate it from/to RFC-2822 format.

The following code snippet show various manners of using the
{\vcode vmime::datetime} object:

\begin{lstlisting}[caption={Using {\vcode vmime::datetime} object}]
// Creating from string in RFC-2822 format
vmime::datetime d1("Sat, 08 Oct 2005 14:07:52 +0200");

// Creating from components
vmime::datetime d2(
   /* date */ 2005, vmime::datetime::OCTOBER, 8,
   /* time */ 14, 7, 52,
   /* zone */ vmime::datetime::GMT2);

// Getting day of week
const int dow = d2.getWeekDay();  // 'dow' should be datetime::SATURDAY
\end{lstlisting}

\subsection{Media type} % ----------------------------------------------------

In MIME, the nature of the data contained in parts is identified using a
media type. A general type (eg. \emph{image}) and a sub-type (eg. \emph{jpeg})
are put together to form a media type (eg. \emph{image/jpeg}). This is also
called the MIME type.

There are a lot of media types officially registered, and vendor-specific
types are possible (they start with ``x-'', eg.
\emph{application/x-zip-compressed}).

In VMime, the object {\vcode vmime::mediaType} represents a media type. There
are also some constants for top-level types and sub-types in the
{\vcode vmime::mediaTypes} namespace. For example, you can instanciate a new
media type with:

\begin{lstlisting}
vmime::mediaType theType(
   /* top-level type */ vmime::mediaTypes::IMAGE,
   /* sub-type */       vmime::mediaTypes::IMAGE_JPEG);

// theType.getType() is "image"
// theType.getSubType() is "jpeg"
// theType.generate() returns "image/jpeg"
\end{lstlisting}

For more information about media types, see
RFC-2046\footnote{http://www.faqs.org/rfcs/rfc2046.html}.

\subsection{Mailbox and mailbox groups} % ------------------------------------

VMime provides several objects for working with mailboxes and addresses.

The {\vcode vmime::address} class is an abstract type for representing an
address: it can be either a mailbox (type {\vcode vmime::mailbox}) or a
mailbox group (type {\vcode vmime::mailboxGroup}). A mailbox is composed of
an email address (mandatory) and possibly a name. A mailbox group is simply
a named list of mailboxes (see Figure \ref{uml_addr_mbox_mboxgroup}).

\begin{lstlisting}[caption={Using mailboxes and mailbox groups}]
vmime::ref <vmime::mailbox> mbox1 = vmime::create <vmime::mailbox>
    (/* name */ vmime::text("John Doe"), /* email */ "john.doe@acme.com");
vmime::ref <vmime::mailbox> mbox2 = vmime::create <vmime::mailbox>
    (/* no name, email only */ "bill@acme.com");

vmime::ref <vmime::mailboxGroup> grp = vmime::create <vmime::mailboxGroup>();
grp->appendMailbox(mbox1);
grp->appendMailbox(mbox2);
\end{lstlisting}

\begin{figure}[ht!]
	\center\includegraphics[width=0.7\textwidth]
		{images/address-mailbox-mailboxgroup.png}
	\caption{Diagram for address-related classes}
	\label{uml_addr_mbox_mboxgroup}
\end{figure}


% ============================================================================
\section{Message, body parts and header}

\subsection{Introduction to MIME messages} % ---------------------------------

A MIME message is a recursive structure in which each part can contains one
or more parts (or \emph{entities}). Each part is composed of a header and
a body (actual contents). Figure \ref{uml_msg_body_header} shows how this
model is implemented in VMime, and all classes that take part in it.

\begin{figure}
	\center\includegraphics[width=1.0\textwidth]
		{images/message-body-header.png}
	\caption{Overall structure of MIME messages}
	\label{uml_msg_body_header}
\end{figure}


\subsection{Header and header fields} % --------------------------------------

\subsubsection{Standard header fields} % .....................................

Header fields carry information about a message (or a part) and its contents.
Each header field has a name and a value. All types that can be used as a
field value inherit from the {\vcode headerFieldValue} class.

You cannot instanciate header fields directly using their constructor.
Instead, you should use the {\vcode headerFieldFactory} object. This ensures
the right field type and value type is used for the specified field name.
For more information about how to use header fields and the factory, see
section \ref{msg-building-simple-message}.

Some standard fields are officially registered and have their value type
specified in a RFC. Table \ref{standard-fields} lists all the fields
registered by default in VMime and the value type they contains.

By default, all unregistered fields have a value of type {\vcode text}.

\begin{table}[!ht]
\begin{center}
\noindent\begin{tabularx}{0.85\textwidth}{|X|X|}
\hline
	{\bf Field Name} &
	{\bf Value Type} \\
\hline
\hline
From & mailbox \\
To & addressList \\
Cc & addressList \\
Bcc & addressList \\
Sender & mailbox \\
Date & datetime \\
Received & relay \\
Subject & text \\
Reply-To & mailbox \\
Delivered-To & mailbox \\
Organization & text \\
Return-Path & path \\
Mime-Version & text \\
Content-Type & mediaType \\
Content-Transfer-Encoding & encoding \\
Content-Description & text \\
Content-Disposition & contentDisposition \\
Content-Id & messageId \\
Content-Location & text \\
Message-Id & messageId \\
In-Reply-To & messageIdSequence \\
References & messageIdSequence \\
Original-Message-Id & messageId \\
Disposition & disposition \\
Disposition-Notification-To & mailboxList \\
\hline
\end{tabularx}
\end{center}
\label{standard-fields}
\caption{Standard fields and their types}
\end{table}


\subsubsection{Parameterized fields} % .......................................

In addition to a value, some header fields can contain one or more
\emph{name=value} couples which are called \emph{parameters}. For example,
this is used in the \emph{Content-Type} field to give more information about
the content:

\begin{verbatim}
   Content-Type: text/plain; charset="utf-8"
\end{verbatim}

Fields that support parameters inherit from the
{\vcode parameterizedHeaderField} class which provides methods to deal with
these parameters: {\vcode appendParameter()}, {\vcode getParameterAt()}...

A parameter is identified by a name (eg. \emph{charset}) and associated to
a value of type {\vcode vmime::text}. Parameters provide helper functions to
convert automatically from basic types to text, and \emph{vice versa}. The
following example illustrates it:

\begin{lstlisting}[caption={Getting and setting parameter value in fields}]
vmime::ref <vmime::parameterizedField> field =
   header->findField("X-Field-That-Contains-Parameters")
      .dynamicCast <vmime::parameterizedField>();

// Use setValue() to convert from a basic type to 'text'
vmime::ref <vmime::parameter> prm = field->getParameter("my-date-param");
prm->setValue(vmime::datetime::now());

// Use getValueAs() to convert from 'text' to a basic type
prm = field->getParameter("my-charset-param");
const vmime::charset ch = prm->getValueAs <vmime::charset>();
\end{lstlisting}

Some fields provide easy access to their standard parameters (see
Table \ref{standard-prm-fields}). This avoids finding the parameter and
\emph{dynamic-casting} its value to the right type. The following code
illustrates how to use it:

\begin{lstlisting}
vmime::ref <vmime::contentTypeField> field =
   header->getField(vmime::fields::CONTENT_TYPE)
      .dynamicCast <vmime::contentTypeField>();

// 1. First solution: the "hard" way
vmime::ref <vmime::parameter> prm = field->findParameter("charset");
const charset ch1 = prm->getValueAs <vmime::charset>();

// 2. Second solution: the simple way
const charset ch2 = field->getCharset();
\end{lstlisting}

\vnote{In both cases, an exception {\vcode no\_such\_parameter} can be
thrown if the parameter does not exist, so be sure to catch it.}

\begin{table}[ht!]
\begin{center}
\noindent\begin{tabularx}{0.85\textwidth}{|l|l|X|}
\hline
	{\bf Field Name} &
	{\bf Field Type} &
	{\bf Parameters} \\
\hline
\hline
Content-Type & contentTypeField & boundary, charset, report-type  \\
\hline
Content-Disposition & contentDispositionField & creation-date,
modification-date, read-date, filename, size \\
\hline
\end{tabularx}
\end{center}
\label{standard-prm-fields}
\caption{Standard parameterized fields}
\end{table}



% ============================================================================
\section{Streams}

\subsection{Streams and stream adapters} % -----------------------------------

Streams permit reading or writing data whatever the underlying system is:
a file on a hard disk, a socket connected to a remote service...

There are two types of streams: input streams (from which you can read data)
and output streams (in which you can write data). Some adapters are provided
for compatibility and convenience, for example:

\begin{itemize}
\item {\vcode inputStreamAdapter} and {\vcode outputStreamAdapter}: allow
to use standard C++ iostreams with VMime;
\item {\vcode inputStreamStringAdapter} and
{\vcode outputStreamStringAdapter}: use a {\vcode vmime::string} object to
read/write data.
\end{itemize}

The following example shows two ways of writing the current date to the
standard output, using stream adapters:

\begin{lstlisting}[caption={Using stream adapters}]
// Get current date and time
const vmime::datetime date = vmime::datetime::now();

// 1. Using outputStreamAdapter
vmime::utility::outputStreamAdapter out(std::cout);

std::cout << "Current date is: ";
date.generate(out);
std::cout << std::endl;

// 2. Using outputStreamStringAdapter
vmime::string dateStr;
vmime::utility::outputStreamStringAdapter outStr(dateStr);

std::cout << "Current date is: " << dateStr << std::endl;
\end{lstlisting}


\subsection{Stream filters} % ------------------------------------------------

Input and output streams can be filtered to perform inline conversions (for
example, there is a filter to convert ``{\textbackslash}r{\textbackslash}n''
sequences to ``{\textbackslash}n''). They inherit from
{\vcode vmime::utility::filteredInputStream} or
{\vcode vmime::utility::filteredOutputStream} and are used like adapters (some
filters also accept parameters; read the documentation).

The most useful filter in VMime (and probably the only one you will need) is
the {\vcode charsetFilteredOutputStream}, which performs inline conversion
of charsets. See \ref{section_charsets} to know how to use it.

\vnote{After you have finished to use a filtered output stream, it is
important to call {\vcode flush()} on it to flush the internal buffer.
If {\vcode flush()} is not called, not all data may be written to the
underlying stream.}


% ============================================================================
\section{Content handlers}

\subsection{Introduction} % --------------------------------------------------

Content handlers are an abstraction for data sources. They are currently used
when some data need to be stored for later use (eg. body part contents,
attachment data, ...). Data can be stored encoded or unencoded (for more
information about encodings, see \ref{section_encodings}).

\subsection{Extracting data from content handlers} % -------------------------

You can extract data in a content handler using the {\vcode extract()} method
(which automatically decodes data if encoded) or {\vcode extractRaw()} (which
extracts data without perfoming any decoding).

The following example shows how to extract the body text from a message, and
writing it to the standard output with charset conversion:

\begin{lstlisting}[caption={Using content handlers to extract body text from
a message}]
// Suppose we already have a message
vmime::ref <vmime::message> msg;

// Obtains a reference to the body contents
vmime::ref <vmime::body> body = msg->getBody();
vmime::ref <vmime::contentHandler> cts = body->getContents();

vmime::utility::outputStreamAdapter out(std::cout);
cts->extract(out);
\end{lstlisting}

\vnote{The body contents is extracted ``as is''. No charset conversion is
performed. See \ref{section_charsets} to know more about conversion between
charsets.}


\subsection{Creating content handlers} % -------------------------------------

When you are building a message, you may need to instanciate content handlers
if you want to set the contents of a body part. The following code snippet
shows how to set the body text of a part from a string:

\begin{lstlisting}[caption={Setting the contents of a body part}]
vmime::ref <vmime::bodyPart> part;  // suppose we have a body part

// Create a new content handler from a string
vmime::ref <vmime::contentHandler> cth =
   vmime::create <vmime::stringContentHandler>("Put body contents here");

// Set the contents
part->getBody()->setContents(cth);
\end{lstlisting}

Content handlers are also used when creating attachments. The following
example illustrates how to create an attachment from a file:

\begin{lstlisting}[caption={Creating an attachment from a file}]
// Create a stream from a file
std::ifstream* fileStream = new std::ifstream();

fileStream->open("/home/vincent/paris.jpg", std::ios::binary);

if (!*fileStream)
   // handle error

vmime::ref <utility::stream> dataStream =
   vmime::create <vmime::utility::inputStreamPointerAdapter>(fileStream);

   // NOTE: 'fileStream' will be automatically deleted
   // when 'dataStream' is deleted

// Create a new content handler
vmime::ref <contentHandler> data =
   vmime::create <vmime::streamContentHandler>(dataStream, 0);

// Now create the attachment
ref <vmime::attachment> att = vmime::create <vmime::defaultAttachment>
   (
   	/* attachment data */ data,
	/* content type */    vmime::mediaType("image/jpeg"),
	/* description */     vmime::text("Holiday photo"),
	/* filename */        vmime::word("paris.jpg")
   );
\end{lstlisting}

You will see later that the {\vcode vmime::fileAttachment} class already
encapsulates all the mechanics to create an attachment from a file.


% ============================================================================
\section{Character sets, charsets and conversions\label{section_charsets}}

Quoting from RFC-2278: \emph{`` The term 'charset' is used to refer to a
method of converting a sequence of octets into a sequence of characters.''}

With the {\vcode vmime::charset} object, VMime supports conversion between
charsets using the {\em iconv} library, which is available on almost all
existing platforms. See {\vcode vmime::charset} and
{\vcode vmime::charsetConverter} in the class documentation to know more
about charset conversion.

The following example shows how to convert data in one charset to another
charset. The data is extracted from the body of a message and converted
to UTF-8 charset:

\begin{lstlisting}[caption={Extracting and converting body contents to a
specified charset}]
vmime::ref <vmime::message> msg;  // we have a message

// Obtain the content handler first
vmime::ref <vmime::body> body = msg->getBody();
vmime::ref <const vmime::contentHandler> cth = body->getContents();

// Then, extract and convert the contents
vmime::utility::outputStreamAdapter out(std::cout);
vmime::utility::charsetFilteredOutputStream fout
   (/* source charset */ body->getCharset(),
    /* dest charset */   vmime::charset("utf-8"),
    /* dest stream */    out);

cth->extract(fout);

fout.flush();  // Very important!
\end{lstlisting}


% ============================================================================
\section{Non-ASCII text in header fields}

MIME standard defines methods\footnote{See RFC-2047: Message Header Extensions
for Non-ASCII Text} for dealing with data which is not 7-bit only (ie. the
ASCII character set), in particular in header fields. For example, the field
``Subject:'' use this data type.

VMime is fully compatible with RFC-2047 and provides two objects for
manipulating 8-bit data: {\vcode vmime::text} and {\vcode vmime::word}. A word
represents textual information encoded in a specified charset. A text is
composed of one or more words.

RFC-2047 describes the process of encoding 8-bit data into a 7-bit form;
basically, it relies on Base64 and Quoted-Printable encoding. Hopefully, all
the encoding/decoding process is done internally by VMime, so creating text
objects is fairly simple:

\begin{lstlisting}[caption={Creating \vcode{vmime::text} objects}]
vmime::string inText = "Linux dans un t�l�phone mobile";
vmime::charset inCharset = "utf-8";

vmime::text outText;
outText.createFromString(inText, inCharset);

// 'outText' now contains 3 words:
//    . <us-ascii>   "Linux dans un "
//    . <utf-8>      "t�l�phone "
//    . <us-ascii>   "mobile"

vmime::ref <vmime::header> header = myMessage->getHeader();
header->Subject()->setValue(outText);
\end{lstlisting}

In general, you will not need to decode RFC-2047-encoded data as the process
is totally transparent in VMime. If you really have to, you can use the
{\vcode vmime::text::decodeAndUnfold()} static method to create a text object
from encoded data.

For example, say you have the following encoded data:

\begin{verbatim}
   Linux dans un =?UTF-8?B?dMOpbMOpcGhvbmUgbW9iaWxl?=
\end{verbatim}

You can simply decode it using the following code:

\begin{lstlisting}[caption={Decoding RFC-2047-encoded data}]
vmime::string inData =
    "Linux dans un =?UTF-8?B?dMOpbMOpcGhvbmUgbW9iaWxl?=";

vmime::text outText;
vmime::text::decodeAndUnfold(inData, &outText);
\end{lstlisting}

{\vcode vmime::text} also provides a function to convert all the words to
another charset in a single call. The following example shows how to convert
text stored in the Subject field of a message:

\begin{lstlisting}[caption={Converting data in a {\vcode vmime::text} to a
specified charset}]
vmime::ref <vmime::message> msg;  // we have a message

vmime::text subject = msg->getHeader()->Subject()->getValue();

const vmime::string subjectText =
   subject.getConvertedText(vmime::charset("utf-8"));

// 'subjectText' now contains the subject in UTF-8 encoding
\end{lstlisting}


% ============================================================================
\section{Encodings\label{section_encodings}}

\subsection{Introduction} % --------------------------------------------------

The MIME standard defines a certain number of encodings to allow data
to be safely transmitted from one peer to another. VMime provides
data encoding and decoding using the {\vcode vmime::utility::encoder::encoder} object.

You should not need to use encoders directly, as all encoding/decoding
process is handled internally by the library, but it is good to know
they exist and how they work.

\subsection{Using encoders} % ------------------------------------------------

You can create an instance of an encoder using the 'vmime::utility::encoder::encoderFactory'
object, giving the encoding name ({\it base64}, {\it quoted-printable}, ...).
The following example creates an instance of the Base64 encoder to encode
some data:

\begin{lstlisting}[caption={A simple example of using an encoder}]
vmime::ref <vmime::utility::encoder::encoder> enc =
    vmime::utility::encoder::encoderFactory::getInstance()->create("base64");

vmime::string inString("Some data to encode");
vmime::utility::inputStreamStringAdapter in(inString);

vmime::string outString;
vmime::utility::outputStreamStringAdapter out(outString);

enc->encode(in, out);

std::cout << "Encoded data is:"  << outString << std::endl;
\end{lstlisting}

\subsection{Enumerating available encoders} % --------------------------------

The behaviour of the encoders can be configured using properties. However,
not all encoders support properties. The following example\footnote{This is
an excerpt from {\vexample example6}} enumerates available encoders and the
supported properties for each of them:

\begin{lstlisting}[caption={Enumerating encoders and their properties}]
vmime::utility::encoder::encoderFactory* ef =
   vmime::utility::encoder::encoderFactory::getInstance();

std::cout << "Available encoders:" << std::endl;

for (int i = 0 ; i < ef->getEncoderCount() ; ++i)
{
   // Output encoder name
   vmime::ref <const vmime::utility::encoder::encoderFactory::registeredEncoder>
      enc = ef->getEncoderAt(i);

   std::cout << "  * " << enc->getName() << std::endl;

   // Create an instance of the encoder to get its properties
   vmime::ref <vmime::utility::encoder::encoder> e = enc->create();

   std::vector <vmime::string> props = e->getAvailableProperties();
   std::vector <vmime::string>::const_iterator it;

   for (it = props.begin() ; it != props.end() ; ++it)
      std::cout << "      - " << *it << std::endl;
\end{lstlisting}


% ============================================================================
\section{Progress listeners}

Progress listeners are used with objects that can notify you about the state
of progress when they are performing an operation.

The {\vcode vmime::utility::progressListener} interface is rather simple:

\begin{lstlisting}
const bool cancel() const;

void start(const int predictedTotal);
void progress(const int current, const int currentTotal);
void stop(const int total);
\end{lstlisting}

{\vcode start()} and {\vcode stop()} are called at the beginning and the end
of the operation, respectively. {\vcode progress()} is called each time the
status of progress changes (eg. a chunk of data has been processed). There is
no unit specified for the values passed in argument. It depends on the
notifier: it can be bytes, percent, number of messages...

Your listener can return {\vcode} in the {\vcode cancel()} method to cancel
the operation. However, be warned that cancelling is not always supported by
the source object.

